\documentclass[12pt, a4paper]{report}
\usepackage[a4paper, total={6in, 8in}]{geometry}
\usepackage[utf8]{inputenc}
\usepackage{multicol}
\setlength{\columnsep}{1cm}
\usepackage{graphicx}
\usepackage{textcomp}
\usepackage{algorithm}
\usepackage{algorithmic}
\usepackage{hyperref}
\hypersetup{
    colorlinks=true,
    linkcolor=blue,
    filecolor=magenta,      
    urlcolor=cyan,
}
% \begin{document}

\usepackage{listings}
\usepackage{color}

\definecolor{dkgreen}{rgb}{0,0.6,0}
\definecolor{gray}{rgb}{0.5,0.5,0.5}
\definecolor{mauve}{rgb}{0.58,0,0.82}

\lstset{frame=tb,
  language=C++,
  aboveskip=3mm,
  belowskip=3mm,
  showstringspaces=false,
  columns=flexible,
  basicstyle={\small\ttfamily},
  numbers=none,
  numberstyle=\tiny\color{gray},
  keywordstyle=\color{blue},
  commentstyle=\color{dkgreen},
  stringstyle=\color{mauve},
  breaklines=true,
  breakatwhitespace=true,
  tabsize=3
}

\title{DAA Assignment 4}
\author{GROUP 17\\
PRAVALLIKA KODI
 (IIT2019234)\\ 
VISHWAM SHRIRAM MUNDADA
 (IIT2019235)\\ 
NOONSAVATH SRAVANA SAMYUKTA
 (IIT2019236)\\
B.tech Information Technology business Informatics\\
Indian Institute of Information Technology, Allahabad}
\date{4 April 2021}


\begin{document}

\maketitle

\begin{multicols}{2}
\section{Question}

\textbf{Given a 2D array, find the maximum sum subarray in it.}





\section{Introduction}
Here, different approaches are analysed and used to achieve results.
These approaches are:\\
1) Brute force\\
2) Using Dynamic programming
 


In this report we will explain our solution approach. We explain our code in detail. We will discuss the time complexity analysis and the space complexity analysis. And last but not least, the conclusion.



\section{Code Explanation}
\subsection{Brute force}

▸In the brute force approach we try to check every possible rectangle in the given
n X m 2D array(where n,m are number of rows and columns respectively).
▸Set the position of the top-left and bottom-right corners of the sub-rectangle and
adding the integers within it while iterating through all the rows sequentially.
▸Parallelly we try to find the maximum subarray sum value.\\
\hline\\
\textbf{Algorithm 1 Brute force}

\begin{lstlisting}

int A[101][101]

function MAIN()
	maxSum <- INT_MIN
	tempSum <- 0
	x <- 0, y <- 0, z <- 0, w <- 0
	n <- 0, m <- 0
	input n, m

	for i <- 0 to n-1
		for j <- 0 to m-1
			input A[i][j]
		end
	end

	for i <- 0 to n-1
		for j <- 0 to m-1
			for k <- 0 to n-1
				for l <- 0 to m-1
					tempSum = FINDSUM(i, j, k, l)

					if tempSum > maxSum
						x <- i
						y <- j
						z <- k
						w <- l
						maxSum = tempSum
					end if
				end
			end
		end
	end

	print x, y, z, w, maxSum


\end{lstlisting}
\hline\\
\textbf{Algorithm 2 Maxsum Pseudo Code}

\begin{lstlisting}
function FINDSUM(x, y, z, w)
	sum <- 0

	for i <- x to z
		for j <- y to w
			sum <- sum + A[i][j]
		end
	end

	return sum

\end{lstlisting}


\subsection{Dynamic programming}


The idea is to fix the left and right columns one by one and find the maximum sum
contiguous rows for every left and right column pair.\\
▸We basically find top and bottom row numbers (which have maximum sum) for every
fixed left and right column pair. To find the top and bottom row numbers, calculate the sum of elements in every row from left to right and store these sums in an array say temp[].\\
▸temp[i] indicates sum of elements from left to right in row i. If we apply Kadane’s 1D
algorithm on temp[], and get the maximum sum subarray of temp, this maximum sum
would be the maximum possible sum with left and right as boundary columns. To get
the overall maximum sum, we compare this sum with the maximum sum so far\\
We use Kadane’s algorithm to reduce the time complexity to O(n^2 x m).\\
\hline\\

\textbf{Algorithm 3 Dynamic programming }
\begin{lstlisting}

int A[101][101]

function MAIN()
	r <- 0, c <- 0, maxSum <- INT_MIN
	x <- 0, y <- 0, z <- 0, w <- 0
	input r, c

	for i <- 0 to r-1
		for j <- 0 to c-1
			input A[i][j]
		end
	end

	for i <- 0 to r-1
		vector<int> sum(c)
		for j <- 0 to r-1
			for col <- 0 to c-1
				sum <- sum + A[j][col]
			end

			vector<int> res <- KADANE(sum)
			if maxSum < res[2]
				x <- i
				y <- res[0]
				z <- j
				w <- res[1]
				maxSum = res[2]
			end if
		end
	end

	print x, y, z, w, maxSum
	
	
function KADANE(V)
	maxSum <- INT_MIN, tempSum <- 0
	st <- -1, end <- -1, localSt <- 0

	for i <- 0 to length[V]
		tempSum <- tempSum + V[i]
		if maxSum < tempSum
			st <- localSt
			end <- i
			maxSum <- tempSum
		end if

		if tempSum < 0
			localSt <- i+1
			tempSum <- 0
		end if
	end

	vector<int> res <- {st, end, maxSum}
	return res


\end{lstlisting}


\section{Algorithm  Analysis}

\subsection{Brute force approach}
In the brute force approach we try to check every possible rectangle in the given
n X m 2D array(where n,m are number of rows and columns respectively).
\\
This solution requires 6 nested loops :\\
2 for the summation of the sub-matrix O(n x m)\\

4 for start and end coordinate of the 2 axis O(n^2 x m^2)\\

The overall time complexity is O(n^3 x m^3)\\



\subsection{Dynamic programming}
Basically, Dynamic algorithm (complexity: O(n)) is used inside a naive
maximum sum subarray problem (complexity: O(n x m)).\\
This gives a total complexity of O(n^2 x m)

\section{Graph analysis}


\textbf{Brute force}\\
In the brute force approach as n increases, the time increases by n^6.\\
\includegraphics[width=\linewidth]{ppl.jpeg}

\textbf{Dynamic programming}\\
In the Dynamic programming approach as n increases, the time increases by n^3.\\

\includegraphics[width=\linewidth]{Screenshot (14).png}


\section{Conclusion}
From the experimental study we concluded that the average
running time of dynamic algorithm is best, which can
be observed from the mutual graph of dynamic
algorithm and Brute Force algorithm as shown.


\section{References}
\parbox{8 cm}{\sloppy
\url{https://www.geeksforgeeks.org/maximum-sum-rectangle-in-a-2d-matrix-dp-27/}

\url{https://www.geeksforgeeks.org/largest-sum-contiguous-subarray/}
}

\section{code}

\url{https://ideone.com/W0rVpg}\\
\url{https://ideone.com/Zw0Ape}



\end{multicols}

\end{document}